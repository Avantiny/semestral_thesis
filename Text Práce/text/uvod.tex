\chapter*{Úvod}
\phantomsection
\addcontentsline{toc}{chapter}{Úvod}
Webové aplikace jsou součástí každodenního života většiny moderní populace a slouží k nejrůznějším účelům od přehrávání videí až po nakupování v e-shopech. Tento stále se rozvíjející obor se díky ohromné poptávce ve věku internetu stále rozčleňuje na další a další specializace, vytváří nové pracovní příležitosti a s tím i důvody, proč se oborem zajímat jakožto kariérní možností. Tato práce se povrchově věnuje všem oblastem vývoje webové aplikace pro účel zpracování obrazových a audiovizuálních děl do formy interaktivního archivu, ve kterém je možné dle různých kritérií filtrovat výsledky vyhledávání. Aby bylo možné tato díla převést do webové aplikace, je nutné je systematicky někam ukládat jako data, která jsou následně zpracována a vyobrazena uživateli. Celý tento proces je možné řešit mnoha způsoby, z nichž jeden je popsán právě v této semestrální práci.